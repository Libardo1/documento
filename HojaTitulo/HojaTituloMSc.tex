%\newpage
%\setcounter{page}{1}
\begin{center}
\begin{figure}
\centering%
\epsfig{file=HojaTitulo/EscudoUN.eps,scale=1}%
\end{figure}
\thispagestyle{empty} \vspace*{2.0cm} \textbf{\huge
Din\'amica estelar y perfiles gal\'acticos de lentes gravitacionales}\\[6.0cm]
%T\'{\i}tulo de la tesis  o trabajo de investigaci\'{o}n}\\[6.0cm]
\Large\textbf{Andr\'es Felipe Granados Cruz}\\[6.0cm]
\small Universidad Nacional de Colombia\\
Facultad, Departamento de F\'isica\\
Ciudad, Colombia\\
2018\\
\end{center}

\newpage{\pagestyle{empty}\cleardoublepage}

\newpage
\begin{center}
\thispagestyle{empty} \vspace*{0cm} \textbf{\huge
Din\'amica estelar y perfiles gal\'acticos de lentes gravitacionales}\\[3.0cm]
\Large\textbf{Andr\'es Felipe Granados Cruz}\\[3.0cm]
\small Tesis o trabajo de grado presentada(o) como requisito parcial para optar al
t\'{\i}tulo de:\\
\textbf{F\'isico}\\[2.5cm]
Director(a):\\
Ph.D. Leonardo Castañeda\\[2.0cm]
L\'{\i}nea de Investigaci\'{o}n:\\
Dinámica galáctica\\
Grupo de Investigaci\'{o}n:\\
Astronomía Galáctica, Gravitación y Cosmología\\[2.5cm]
Universidad Nacional de Colombia\\
Facultad de Ciencias, Departamento de Física\\
Bogotá D.C., Colombia\\
2018\\
\end{center}

\newpage{\pagestyle{empty}\cleardoublepage}

\newpage
\thispagestyle{empty} \textbf{}\normalsize
\\\\\\%
\textbf{(Dedicatoria o un lema)}\\[4.0cm]

\begin{flushright}
\begin{minipage}{8cm}
    \noindent
        \small
        Su uso es opcional y cada autor podr\'{a} determinar la distribuci\'{o}n del texto en la p\'{a}gina, se sugiere esta presentaci\'{o}n. En ella el autor dedica su trabajo en forma especial a personas y/o entidades.\\[1.0cm]\\
        Por ejemplo:\\[1.0cm]
        A mis padres\\[1.0cm]\\
        o\\[1.0cm]
        La preocupaci\'{o}n por el hombre y su destino siempre debe ser el
        inter\'{e}s primordial de todo esfuerzo t\'{e}cnico. Nunca olvides esto
        entre tus diagramas y ecuaciones.\\\\
        Albert Einstein\\
\end{minipage}
\end{flushright}

\newpage{\pagestyle{empty}\cleardoublepage}

\newpage
\thispagestyle{empty} \textbf{}\normalsize
\\\\\\%
\textbf{\LARGE Agradecimientos}
\addcontentsline{toc}{chapter}{\numberline{}Agradecimientos}\\\\
Esta secci\'{o}n es opcional, en ella el autor agradece a las personas o instituciones que colaboraron en la realizaci\'{o}n de la tesis  o trabajo de investigaci\'{o}n. Si se incluye esta secci\'{o}n, deben aparecer los nombres completos, los cargos y su aporte al documento.\\

\newpage{\pagestyle{empty}\cleardoublepage}

\newpage
\textbf{\LARGE Resumen}
\addcontentsline{toc}{chapter}{\numberline{}Resumen}\\\\
Un sistema estelar puede ser descrito por una función de distribución de partículas que no colisionan, en la ecuación de Vlasov que relaciona el potencial gravitacional del sistema con su función de distribución. El primer y segundo momentos de la función de distribución da cuenta de la velocidad y del tensor de dispersión de velocidades que caracteriza la cinemática como observables del sistema y su relación con la dinámica está dada por las ecuaciones de Jeans. Se revisan los métodos de expansión gaussiano del potencial gravitacional y el modelamiento axialmente simétrico de las ecuaciones de Jeans con parámetro de anisotropía del elipsoide de velocidades. Por otra parte la curva de rotación galáctica permite obtener de manera directa la distribución de materia apoyado por fuerte evidencia observacional gracias al método de descomposición del potencial gravitacional en uno de bulbo, disco estelar y de materia oscura. La descripción que viene de la dinámica estelar es contrastado con el resultado que proviene del efecto de lente gravitacional. Éste último surge de la deflexión a la que es sometida la luz de un objeto fuente lejano, por efecto de la gravedad de otro objeto interpuesto entre la fuente y el observador, denominado lente. Los modelos de lente permiten obtener la posición en el cielo del objeto fuente mediante las posiciones de sus imágenes producidas por la lente y del ángulo de deflexión. El ángulo de deflexión contiene la información del potencial gravitacional deflector y así mismo de la distribución de materia mediante la ecuación de Poisson. El objeto de estudio es el sistema lente+galaxia espiral SDSS J1331+3628 del cual se realizó las deducciones para obtener su distribución de materia por el ajuste de la curva de rotación de la galaxia y por medio de un modelo de lente independiente de escala.\\

El resumen es una presentaci\'{o}n abreviada y precisa (la NTC 1486 de 2008 recomienda revisar la norma ISO 214 de 1976). Se debe usar una extensi\'{o}n m\'{a}xima de 12 renglones. Se recomienda que este resumen sea anal\'{\i}tico, es decir, que sea completo, con informaci\'{o}n cuantitativa y cualitativa, generalmente incluyendo los siguientes aspectos: objetivos, dise\~{n}o, lugar y circunstancias, pacientes (u objetivo del estudio), intervenci\'{o}n, mediciones y principales resultados, y conclusiones. Al final del resumen se deben usar palabras claves tomadas del texto (m\'{\i}nimo 3 y m\'{a}ximo 7 palabras), las cuales permiten la recuperaci\'{o}n de la informaci\'{o}n.\\

\textbf{\small Palabras clave: función de distribución, ecuación de Vlasov, ecuaciones de Jeans, galaxias: cinemática y dinámica, anisotropía, elipsoide de velocidades, lente gravitacional, bulbo estelar, materia oscura  }.\\


\textbf{\LARGE Abstract}\\\\
Es el mismo resumen pero traducido al ingl\'{e}s. Se debe usar una extensi\'{o}n m\'{a}xima de 12 renglones. Al final del Abstract se deben traducir las anteriores palabras claves tomadas del texto (m\'{\i}nimo 3 y m\'{a}ximo 7 palabras), llamadas keywords. Es posible incluir el resumen en otro idioma diferente al espa\~{n}ol o al ingl\'{e}s, si se considera como importante dentro del tema tratado en la investigaci\'{o}n, por ejemplo: un trabajo dedicado a problemas ling\"{u}\'{\i}sticos del mandar\'{\i}n seguramente estar\'{\i}a mejor con un resumen en mandar\'{\i}n.\\[2.0cm]
\textbf{\small Keywords: palabras clave en ingl\'{e}s(m\'{a}ximo 10 palabras, preferiblemente seleccionadas de las listas internacionales que permitan el indizado cruzado)}\\